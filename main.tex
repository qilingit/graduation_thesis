\documentclass[12pt]{article}
\usepackage[square,numbers]{natbib}     % package utilisé pour la partie de référence
\bibliographystyle{plain}
\usepackage[utf8x]{inputenc}
\usepackage[french]{babel}
\usepackage{url}
\usepackage{amsmath}
\usepackage{graphicx}
\graphicspath{{images/}}
\usepackage{parskip}
\usepackage{fancyhdr}
\usepackage{vmargin}
\usepackage[T1]{fontenc}

%Configuration pour les code insérer
\usepackage{listings}
%\usepackage{color}
\lstset{frame=tb,
  language=Java,
  aboveskip=3mm,
  belowskip=3mm,
  showstringspaces=false,
  columns=flexible,
  basicstyle={\small\ttfamily},
  numbers=none,
  numberstyle=\tiny\color{gray},
  keywordstyle=\color{blue},
  commentstyle=\color{dkgreen},
  stringstyle=\color{black},
  breaklines=true,
  breakatwhitespace=true,
  tabsize=3
}

%\setcounter{secnumdepth}{4}
%\titleformat{\paragraph}

% utiliser pour faire annexe
\usepackage{appendix}

% make a clibable table of content
\usepackage[colorlinks=false]{hyperref}

%Includes "References" in the table of contents
\usepackage[nottoc]{tocbibind}
\usepackage{float} % utiliser cet package pour insérer les figures, permet de mettre juste après le texte

% Ce package est pour définir les couleurs de texte
\usepackage{xcolor}

\usepackage{vmargin}
%1 est la marge gauche
%2 est la marge en haut
%3 est la marge droite
%4 est la marge en bas
%5 fixe la hauteur de l’entête
%6 fixe la distance entre l’entête et le texte
%7 fixe la hauteur du pied de page
%8 fixe la distance entre le texte et le pied de page
\setmarginsrb{3 cm}{2.5 cm}{3 cm}{2.5 cm}{2 cm}{1.5 cm}{1 cm}{1.5 cm}

\title{\textbf{Rapport de projet de fin d'études\newline Automatisation de Test fonctionnel}}    % Title
\author{ZHANG Qilin}		% Author
\date{Septembre 2019}											% Date

\makeatletter
\let\thetitle\@title
\let\theauthor\@author
\let\thedate\@date
\makeatother

\pagestyle{fancy}
\fancyhf{}
\rhead{\leftmark}
\lhead{Sorbonne Université Master 2 STL\newline \theauthor}
\usepackage{lastpage}
\cfoot{Page \thepage \ sur\ \pageref{LastPage}}
\renewcommand{\headrulewidth}{2pt}
\renewcommand{\footrulewidth}{1pt}

\begin{document}

%%%%%%%%%%%%%%%%%%%%%%%%%%%%%%%%%%%%%%%%%%%%%%%%%%%%%%%%%%%%%%%%%%%%%%%%%%%%%%%%%%%%%%%%%

\begin{titlepage}
	\centering
    \vspace*{0.5 cm}
    \includegraphics[scale = 0.15]{Logo_officiel_Sorbonne_University}\\[0.8 cm]	% University Logo
    \includegraphics[scale = 0.15]{SAP_R_grad.jpg}
    \textsc{\LARGE Sorbonne Université}\\[2.0 cm]	% University Name
	\textsc{\Large Master 2 Science et Technologie du Logiciel\\2018 - 2019}\\[0.5 cm]		% Course Code
	%\textsc{\large Développement d'Application Réticulaire }\\[0.5 cm]	% Course Name
	\rule{\linewidth}{0.8 mm} \\[0.6 cm]
	{ \huge \bfseries \thetitle}\\
	\rule{\linewidth}{0.8 mm} \\[2.6 cm]

	\begin{minipage}{0.4\textwidth}
		\begin{flushleft} \large
			\emph{Etudiants:}\\
			\theauthor
			\end{flushleft}
			\end{minipage}~
			\begin{minipage}{0.4\textwidth}
			\begin{flushright} \large
			\emph{Tuteurs:} \\
			Camilla CHRISTIANSEN, Sebastien COUDRAY % Your Student Number
		\end{flushright}
	\end{minipage}\\[0.5 cm]
	
	{\large \thedate}\\[2 cm]
 
	\vfill
	
\end{titlepage}

%%%%%%%%%%%%%%%%%%%%%%%%%%%%%%%%%%%%%%%%%%%%%%%%%%%%%%%%%%%%%%%%%%%%%%%%%%%%%%%%%%%%%%%%%
\setcounter{tocdepth}{2} % Set the depth of table of contents
\tableofcontents
\pagebreak

%%%%%%%%%%%%%%%%%%%%%%%%%%%%%%%%%%%%%%%%%%%%%%%%%%%%%%%%%%%%%%%%%%%%%%%%%%%%%%%%%%%%%%%
\newpage
\section{Remerciements}

%%%%%%%%%%%%%%%%%%%%%%%%%%%%%%%%%%%%%%%%%%%%%%%%%%%%%%%%%%%%%%%%%%%%%%%%%%%%%%%%%%%%%%%%%
\newpage
\section{Introduction}


%%%%%%%%%%%%%%%%%%%%%%%%%%%%%%%%%%%%%%%%%%%%%%%%%%%%%%%%%%%%%%%%%%%%%%%%%
\newpage
\section{Présentation d'entreprise}
SAP SE (\textit{\textbf{S}ysteme, \textbf{A}nwendungen und \textbf{P}rodukte in der Datenverarbeitung}, "\textbf{S}ystems, \textbf{A}pplications \& \textbf{P}roducts in Data Processing") est une entreprise qui conçoit et vend des logiciels, notamment des systèmes de gestion et de maintenance, principalement à destination des entreprises et des institutions dans le monde entier. SAP est une entreprise internationale et sont siège se trouve à Waldorf, Allemagne, SAP est le premier éditeur de logiciels en Europe et le quatrième dans le monde.

\subsection{Histoire}
    \subsubsection{1972-1980: Les jeunes années}
    En 1972, 5 anciens employés d'IBM - Dietmar Hopp, Hans-Werner Hector, Hasso Plattner, Klaus E. Tschira, et Claus Wellenreuther - fondaient Systems Applications and Products in Data Processing (Systèmes, Applications et Progiciels), à Mannheim, en Allemagne. S'appuyant sur le rêve de l'informatique «temps réel»: un logiciel qui traite les données lorsque les clients en ont besoin plutôt que de passer la nuit à la volée. SAP a délivré la release SAP R/1 en 1973.

    \subsubsection{1981-1990 : Le SAP R/2}
    Le temps réel touche davantage l’activité: les processus d’application logicielle mainframe packagés SAP R/2 intègrent l’ensemble des fonctions commerciales d’une entreprise.

    \subsubsection{1991-2000 : Le SAP R/3}
    Temps réel sur le bureau: une version client-serveur du logiciel d'application standard permet aux entreprises de fonctionner plus efficacement dans le monde entier.
    \subsubsection{2001-2010 : Des données en temps réel où et quand vous en avez besoin}
    Real-time moves to the Web and beyond: Cloud computing, mobile, and in-memory computing open up new horizons for real-time data access – anywhere.
    
    \subsubsection{2011-Présent : Résultats d'enregistrement des supports de prise en charge en mémoire, en nuage et en réseau d'entreprise}
    La croissance continue de l'entreprise est tirée par la plate-forme en mémoire SAP HANA qui permet aux analyses de données ultra-rapides de devenir une réalité. Des acquisitions stratégiques associées à une innovation continue font de SAP un leader des réseaux d’informatique en nuage et du commerce électronique. Avec le lancement de SAP S / 4HANA et de SAP C / 4HANA, SAP dévoile la nouvelle génération de logiciels d'entreprise destinés à aider les clients à devenir des entreprises intelligentes.
    
%%%%%%%%%%%%%%%%%%%%%%%%%%%%%%%%%%%%%%%%%%%%%%%%%%%%%%%%%%%%%%%%%%%%%%%%%
\newpage
\section{Présentation du produit}

%%%%%%%%%%%%%%%%%%%%%%%%%%%%%%%%%%%%%%%%%%%%%%%%%%%%%%%%%%%%%%%%%%%%%%
\newpage
\section{Contenu de mon travail}


%%%%%%%%%%%%%%%%%%%%%%%%%%%%%%%%%%%%%%%%%%%%%%%%%%%%%%%%%%%%%%%%%%%%%%%%%
\section{Conclusion du projet}
\paragraph{}

\newpage
\bibliography{biblist}

\newpage
\section{Annexe}
\newpage

%\begin{center}
%    \includegraphics[width=\textwidth]{maquette_mainPage.png}
%\end{center}

\end{document}