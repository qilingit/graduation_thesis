\documentclass[12pt]{article}
\usepackage{natbib}     % package utilisé pour la partie de référence
\usepackage[french]{babel}
\usepackage{url}
\usepackage[utf8x]{inputenc}
\usepackage{amsmath}
\usepackage{graphicx}
\graphicspath{{images/}}
\usepackage{parskip}
\usepackage{fancyhdr}
\usepackage{vmargin}
\usepackage[T1]{fontenc}

%Configuration pour les code insérer
\usepackage{listings}
%\usepackage{color}
\lstset{frame=tb,
  language=Java,
  aboveskip=3mm,
  belowskip=3mm,
  showstringspaces=false,
  columns=flexible,
  basicstyle={\small\ttfamily},
  numbers=none,
  numberstyle=\tiny\color{gray},
  keywordstyle=\color{blue},
  commentstyle=\color{dkgreen},
  stringstyle=\color{black},
  breaklines=true,
  breakatwhitespace=true,
  tabsize=3
}

%\setcounter{secnumdepth}{4}
%\titleformat{\paragraph}

% utiliser pour faire annexe
\usepackage{appendix}

% make a clibable table of content
\usepackage[colorlinks=false]{hyperref}

%Includes "References" in the table of contents
\usepackage[nottoc]{tocbibind}
\usepackage{float} % utiliser cet package pour insérer les figures, permet de mettre juste après le texte

% Ce package est pour définir les couleurs de texte
\usepackage{xcolor}

%1 est la marge gauche
%2 est la marge en haut
%3 est la marge droite
%4 est la marge en bas
%5 fixe la hauteur de l’entête
%6 fixe la distance entre l’entête et le texte
%7 fixe la hauteur du pied de page
%8 fixe la distance entre le texte et le pied de page
\setmarginsrb{3 cm}{2.5 cm}{3 cm}{2.5 cm}{2 cm}{1.5 cm}{1 cm}{1.5 cm}

\title{\textbf{Rapport de projet de fin d'études :\newline \newline Automatisation de Test fonctionnel}}    % Title
\author{ZHANG Qilin}		% Author
\date{Semptembre 2019}											% Date

\makeatletter
\let\thetitle\@title
\let\theauthor\@author
\let\thedate\@date
\makeatother

\pagestyle{fancy}
\fancyhf{}
\rhead{\theauthor}
\lhead{\thetitle}
\cfoot{\thepage}

\begin{document}

%%%%%%%%%%%%%%%%%%%%%%%%%%%%%%%%%%%%%%%%%%%%%%%%%%%%%%%%%%%%%%%%%%%%%%%%%%%%%%%%%%%%%%%%%

\begin{titlepage}
	\centering
    \vspace*{0.5 cm}
    \includegraphics[scale = 0.15]{Logo_officiel_Sorbonne_University}\\[0.8 cm]	% University Logo
    \includegraphics[scale = 0.15]{SAP_R_grad.jpg}
    \textsc{\LARGE Sorbonne Université}\\[2.0 cm]	% University Name
	\textsc{\Large Master 2 Science et Technologie du Logiciel\\2018 - 2019}\\[0.5 cm]		% Course Code
	%\textsc{\large Développement d'Application Réticulaire }\\[0.5 cm]	% Course Name
	\rule{\linewidth}{0.2 mm} \\[0.4 cm]
	{ \huge \bfseries \thetitle}\\
	\rule{\linewidth}{0.2 mm} \\[0.4 cm]

	\begin{minipage}{0.4\textwidth}
		\begin{flushleft} \large
			\emph{Etudiants:}\\
			\theauthor
			\end{flushleft}
			\end{minipage}~
			\begin{minipage}{0.4\textwidth}
			\begin{flushright} \large
			\emph{Responsable:} \\
			Romain Demangeon, Marwan Ghanem % Your Student Number
		\end{flushright}
	\end{minipage}\\[0.5 cm]
	
	{\large \thedate}\\[2 cm]
 
	\vfill
	
\end{titlepage}

%%%%%%%%%%%%%%%%%%%%%%%%%%%%%%%%%%%%%%%%%%%%%%%%%%%%%%%%%%%%%%%%%%%%%%%%%%%%%%%%%%%%%%%%%

\tableofcontents
\pagebreak

%%%%%%%%%%%%%%%%%%%%%%%%%%%%%%%%%%%%%%%%%%%%%%%%%%%%%%%%%%%%%%%%%%%%%%%%%%%%%%%%%%%%%%%%%
\newpage
\section{Introduction}


%%%%%%%%%%%%%%%%%%%%Manuel d'utilisation%%%%%%%%%%%%%%%%%%%%%%
\newpage
\section{Manuel d'utilisation}

\subsection{Présentation de l'application}


\section{Conclusion du projet}
\paragraph{}

\newpage
\bibliographystyle{plain}
\bibliography{biblist}


\newpage
\section{Annexe}
\appendix
\section{Maquette faite avec Balsamiq Mockups 3}
\label{Annexe-Maquette}
\href{https://drive.google.com/file/d/13S1_j926OpUDYI9evYTFl-aJDEXmhMbn/view?usp=sharing}{\underline{Lien pour maquette d'application}}
%\begin{center}
%    \includegraphics[width=\textwidth]{maquette_mainPage.png}
%\end{center}

\end{document}